\chapter{\ifenglish Introduction\else บทนำ\fi}

\section{\ifenglish Project rationale\else ที่มาของโครงงาน\fi}
\hspace{2em} ในปัจจุบันโรงงานอุตสาหกรรมจำนวนมากได้ยกระดับกระบวนการผลิตให้มีความเป็นอัตโนมัติสูงขึ้น \\ โดยอาศัยระบบควบคุมอุตสาหกรรม (Industrial Control Systems: ICS) ที่ทำงานร่วมกับระบบเทคโนโลยีปฏิบัติการ (Operational Technology: OT) ซึ่งเชื่อมโยงเซนเซอร์และแอกชูเอเตอร์เข้ากับเครื่องจักรจริง อย่างไรก็ตาม การที่ระบบ OT ถูกเชื่อมต่อเข้ากับเครือข่ายดิจิทัลมากขึ้น ย่อมทำให้ความเสี่ยงจากภัยคุกคามทางไซเบอร์เพิ่มสูงขึ้นตามไปด้วย การโจมตีดังกล่าวอาจทำให้ข้อมูลสูญหายหรือบิดเบือน รวมถึงก่อให้เกิดความเสียหายต่อเครื่องจักร กระบวนการผลิต และความปลอดภัยของบุคลากรได้โดยตรง ดังนั้นโครงงานนี้จึงถูกริเริ่มขึ้นเพื่อพัฒนากลไกการตรวจจับพฤติกรรมที่ผิดปกติในระบบ OT โดยใช้เทคนิคการเรียนรู้ของเครื่อง (Machine Learning) เพื่อเสริมสร้างความปลอดภัยและลดความเสียหายที่อาจเกิดขึ้นในโรงงานอุตสาหกรรม

\section{\ifenglish Objectives\else วัตถุประสงค์ของโครงงาน\fi}
\begin{enumerate}
    \item ออกแบบและพัฒนาโมเดลตรวจจับความผิดปกติจากข้อมูลเซนเซอร์และแอกชูเอเตอร์ในระบบ OT \\ โดยใช้การวิเคราะห์เชิงเวลา (time-series analysis) ร่วมกับโมเดลเชิงลึก เช่น CNN และ LSTM
    \item ระบุเหตุการณ์ที่มีลักษณะเป็นการโจมตีทางไซเบอร์หรือพฤติกรรมที่ผิดปกติในกระบวนการผลิต
    \item พัฒนาต้นแบบ (prototype) ที่สามารถนำไปประยุกต์ใช้กับสภาพแวดล้อมในโรงงานจริงได้
\end{enumerate}

\section{\ifenglish Project scope\else ขอบเขตของโครงงาน\fi}

\subsection{\ifenglish Hardware scope\else ขอบเขตด้านฮาร์ดแวร์\fi}
\begin{enumerate}
    \item โครงงานนี้ ไม่ได้มุ่งเน้นการพัฒนาหรือใช้งานฮาร์ดแวร์จริง เช่น PLC, ปั๊ม, หรือวาล์ว
    \item ส่วนประกอบเหล่านี้ถูก จำลองผ่านข้อมูลจาก SWaT dataset เท่านั้น
    \item การดำเนินงานทั้งหมดจึงมุ่งเน้นที่การวิเคราะห์เชิงข้อมูลและการออกแบบโมเดล Machine Learning โดยไม่เกี่ยวข้องกับการสร้างหรือทดสอบระบบฮาร์ดแวร์จริง
\end{enumerate}

\subsection{\ifenglish Software scope\else ขอบเขตด้านซอฟต์แวร์\fi}

\subsubsection{การเก็บและใช้ข้อมูล (Data Collection)}
\begin{enumerate}
    \item ใช้ชุดข้อมูล SWaT Dataset (Secure Water Treatment) ซึ่งเป็นข้อมูลจากระบบจำลองโรงงานน้ำประปาที่ใช้กันอย่างแพร่หลายในการศึกษาด้านความปลอดภัยของระบบ ICS/OT
    \item ข้อมูลประกอบด้วยค่าการทำงานของเซนเซอร์และแอกชูเอเตอร์ที่มีทั้งสภาวะปกติและสภาวะถูกโจมตี
    \item การนำเข้าข้อมูลจะใช้วิธีการอ่านไฟล์ CSV และเตรียมข้อมูลให้อยู่ในรูปแบบที่เหมาะสมต่อการประมวลผล
\end{enumerate}

\subsubsection{การตรวจจับและจำแนกความผิดปกติ (Threat Detection and Classification) }
\begin{enumerate}
    \item มุ่งเน้นการตรวจจับเหตุการณ์ที่มีลักษณะผิดปกติ เช่น การเปลี่ยนค่าของเซนเซอร์อย่างผิดธรรมชาติ หรือการสั่งการอุปกรณ์ที่ไม่สอดคล้องกับกระบวนการจริง
    \item การจำแนกความผิดปกติออกเป็นสองกลุ่มหลัก ได้แก่
    \begin{enumerate}
        \item เหตุการณ์ปกติ (Normal events)
        \item เหตุการณ์โจมตี/ผิดปกติ (Anomaly/Attack events)
    \end{enumerate}
    \item การสร้าง Label จะอ้างอิงจากช่วงเวลาที่กำหนดไว้ในเอกสาร SWaT dataset
\end{enumerate}

\subsubsection{การประมวลผลและจัดการข้อมูล (Data Implementation)}
\begin{enumerate}
    \item การทำความสะอาดข้อมูล (data preprocessing) เช่น การจัดการค่าที่หายไป, การแทนค่าผิดปกติ, และการ normalize ข้อมูลให้อยู่ในสเกลเดียวกัน
    \item การแบ่งข้อมูลออกเป็น Training set, Validation set และ Test set โดยอ้างอิงตามลำดับเวลาเพื่อเลี่ยงการรั่วไหลของข้อมูล (data leakage)
    \item การสร้าง sliding windows สำหรับข้อมูลเชิงเวลา (time-series) เพื่อเตรียมให้เป็นอินพุตของโมเดล
\end{enumerate}

\subsubsection{การสร้างโมเดลและเปรียบเทียบประสิทธิภาพ (Model Comparison and Performance Evaluation)}
\begin{enumerate}
    \item พัฒนาโมเดลตรวจจับความผิดปกติที่อิงกับ Deep Learning ได้แก่ 
    \begin{enumerate}
        \item Convolutional Neural Network (CNN)
        \item  Long Short-Term Memory (LSTM)
    \end{enumerate}
    \item ทดสอบโมเดลหลายรูปแบบ เช่น CNN 1D สำหรับจับ pattern ตามลำดับเวลา และ LSTM สำหรับตรวจจับ dependency ระหว่างข้อมูลในช่วงยาว
    \item ประเมินผลลัพธ์ด้วยตัวชี้วัดมาตรฐาน
    \begin{enumerate}
        \item Accuracy, Precision
        \item Recall, F1-score
        \item Confusion Matrix 
        \item AUC-PR/ROC
    \end{enumerate}
    \item เปรียบเทียบประสิทธิภาพของแต่ละโมเดลเพื่อหาวิธีที่เหมาะสมที่สุดต่อการใช้งาน
\end{enumerate}

\subsubsection{ผลลัพธ์ที่คาดว่าจะได้รับ (Expected Outcomes)}
\begin{enumerate}
    \item ได้ต้นแบบ (Prototype) ของระบบตรวจจับความผิดปกติในข้อมูลจากระบบ ICS/OT
    \item โมเดลที่พัฒนาแล้วสามารถแยกความแตกต่างระหว่างเหตุการณ์ปกติและเหตุการณ์ผิดปกติได้ในระดับที่แม่นยำ
    \item ได้ชุดข้อมูลและโค้ดที่สามารถนำไปปรับใช้หรือต่อยอดในงานวิจัยด้านความปลอดภัยไซเบอร์ \\ สำหรับระบบอุตสาหกรรม
    \item สนับสนุนการเพิ่มมาตรการด้าน ความมั่นคงปลอดภัยเชิงปฏิบัติการ (Operational Security) ในโรงงานอุตสาหกรรม
\end{enumerate}

\section{\ifenglish Expected outcomes\else ประโยชน์ที่ได้รับ\fi}

\subsubsection{ด้านความรู้และความเข้าใจ}
\begin{enumerate}
    \item ได้ความเข้าใจในเชิงลึกเกี่ยวกับประเด็นด้านความมั่นคงปลอดภัยในระบบควบคุมอุตสาหกรรม (OT/ICS)
    \item เพิ่มพูนทักษะในการนำ Machine Learning มาประยุกต์ใช้กับข้อมูลเชิงเวลา (time-series)
\end{enumerate}

\subsubsection{ด้านการพัฒนาเทคโนโลยี}
\begin{enumerate}
    \item ได้ต้นแบบ (Prototype) ของระบบตรวจจับความผิดปกติที่สามารถนำไปต่อยอดการพัฒนาระบบจริงได้
    \item แสดงให้เห็นความเป็นไปได้ของการนำ AI/ML มาใช้แก้ปัญหาความปลอดภัยในภาคอุตสาหกรรม
\end{enumerate}

\subsubsection{ด้านการประยุกต์ใช้}
\begin{enumerate}
    \item ผลงานที่ได้สามารถนำไปใช้ตรวจจับการโจมตีหรือพฤติกรรมผิดปกติในโรงงานอุตสาหกรรมจริง
    \item ช่วยเพิ่มความปลอดภัยทั้งในด้านข้อมูล เครื่องจักร และบุคลากร ลดความเสี่ยงจากการหยุดชะงักของกระบวนการผลิต
\end{enumerate}

\subsubsection{ด้านการวิจัยและการศึกษา}
\begin{enumerate}
    \item สามารถนำองค์ความรู้และผลลัพธ์ไปต่อยอดในเชิงวิจัยด้านความปลอดภัยไซเบอร์ \\ สำหรับระบบอุตสาหกรรม
    \item เป็นกรณีศึกษาในการบูรณาการ Machine Learning กับความปลอดภัยไซเบอร์ (AI for Cybersecurity)
\end{enumerate}

\section{\ifenglish Technology and tools\else เทคโนโลยีและเครื่องมือที่ใช้\fi}

\subsection{\ifenglish Hardware technology\else เทคโนโลยีด้านฮาร์ดแวร์\fi}

\subsubsection{Graphics Processing Unit (GPU)}
\begin{enumerate}
    \item ใช้ GPU จากบริการ Google Colab เพื่อเร่งความเร็วในการฝึกโมเดลเชิงลึก (Deep Learning)
\end{enumerate}

\subsection{\ifenglish Software technology\else เทคโนโลยีด้านซอฟต์แวร์\fi}

\subsubsection{ภาษาโปรแกรม (Programming Language)}
\begin{enumerate}
    \item ใช้ GPU จากบริการ Google Colab เพื่อเร่งความเร็วในการฝึกโมเดลเชิงลึก (Deep Learning)
\end{enumerate}

\subsubsection{เฟรมเวิร์กและไลบรารีสำหรับ Machine Learning (Frameworks and Libraries)}
\begin{enumerate}
    \item Scikit-learn: สำหรับ preprocessing การสร้าง baseline model และการประเมินผลเบื้องต้น
    \item TensorFlow และ PyTorch: สำหรับการพัฒนาและฝึกโมเดลเชิงลึก (Deep Learning) เช่น CNN และ LSTM
    \item Pandas และ NumPy: สำหรับการจัดการข้อมูลเชิงตารางและการคำนวณเชิงตัวเลข
    \item Matplotlib และ Seaborn: สำหรับการสร้าง Visualization วิเคราะห์ข้อมูลและผลลัพธ์ของโมเดล
\end{enumerate}

\subsubsection{สภาพแวดล้อมการพัฒนา (Development Environment)}
\begin{enumerate}
    \item Jupyter Notebook และ Google Colab: สำหรับการทดลองเชิงโค้ด การประเมินผล และการรันโมเดลบน GPU
    \item GitHub: สำหรับการจัดการซอร์สโค้ดและการทำงานร่วมกัน
\end{enumerate}

\section{\ifenglish Project plan\else แผนการดำเนินงาน\fi}

\begin{plan}{6}{2025}{2}{2026}
    \planitem{7}{2025}{8}{2025}{ศึกษาค้นคว้า}
    \planitem{8}{2025}{1}{2026}{ชิล}
    \planitem{2}{2026}{2}{2026}{เผา}
    \planitem{12}{2025}{1}{2026}{ทดสอบ}
\end{plan}

\section{\ifenglish Roles and responsibilities\else บทบาทและความรับผิดชอบ\fi}
อธิบายว่าในการทำงาน นศ. มีการกำหนดบทบาทและแบ่งหน้าที่งานอย่างไรในการทำงาน จำเป็นต้องใช้ความรู้ใดในการทำงานบ้าง

\section{\ifenglish%
Impacts of this project on society, health, safety, legal, and cultural issues
\else%
ผลกระทบด้านสังคม สุขภาพ ความปลอดภัย กฎหมาย และวัฒนธรรม
\fi}

\hspace{2em} โครงงานนี้มุ่งเน้นการพัฒนาระบบตรวจจับความผิดปกติในระบบควบคุมอุตสาหกรรม (OT) ซึ่งมีผลกระทบในหลายมิติ ดังนี้

\subsubsection{ด้านสังคม}
    \hspace{2em} การเพิ่มประสิทธิภาพในการตรวจจับและป้องกันการโจมตีไซเบอร์ในระบบอุตสาหกรรม สามารถลด \\ ความเสี่ยงของเหตุการณ์ที่อาจกระทบต่อประชาชน เช่น การปนเปื้อนของระบบน้ำในกรณีศึกษา \\ SWaT dataset หรือการหยุดชะงักของการให้บริการสาธารณะ ซึ่งมีผลโดยตรงต่อคุณภาพชีวิตของสังคม

\subsubsection{ด้านสุขภาพและความปลอดภัย}
    \hspace{2em} ระบบ OT ที่ถูกรุกรานสามารถส่งผลกระทบต่อความปลอดภัยของพนักงานและชุมชนโดยรอบ เช่น การทำงานผิดพลาดของปั๊ม วาล์ว หรือเซนเซอร์ในกระบวนการผลิตน้ำ หากตรวจจับและตอบสนองได้เร็ว จะช่วยลดความเสี่ยงต่ออุบัติเหตุและผลกระทบต่อสุขภาพของผู้บริโภคและผู้ปฏิบัติงาน

\subsubsection{ด้านกฎหมายและมาตรฐาน}
    \hspace{2em} อุตสาหกรรมสมัยใหม่ต้องสอดคล้องกับข้อกำหนดด้านความมั่นคงปลอดภัยไซเบอร์ (Cybersecurity Regulations) และมาตรฐานสากล เช่น IEC 62443 และ ISO/IEC 27001 การพัฒนาโครงงานนี้ช่วยสร้างแนวทางที่สอดคล้องกับข้อกำหนดดังกล่าว และสามารถนำไปปรับใช้เพื่อตอบสนองต่อกฎหมาย \\ และมาตรการควบคุมของหน่วยงานที่เกี่ยวข้อง

\subsubsection{ด้านวัฒนธรรมองค์กร}
    \hspace{2em} การนำเทคโนโลยี Machine Learning มาปรับใช้กับระบบ OT ช่วยเสริมสร้างวัฒนธรรมด้าน ความตระหนักรู้ทางไซเบอร์ (Cybersecurity Awareness) ในองค์กร ส่งเสริมให้ผู้ปฏิบัติงานและผู้บริหารเห็นความสำคัญของความปลอดภัยเชิงข้อมูลควบคู่ไปกับความปลอดภัยเชิงกายภาพ