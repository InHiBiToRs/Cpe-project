\chapter{\ifproject%
\ifenglish Experimentation and Results\else การทดลองและผลลัพธ์\fi
\else%
\ifenglish System Evaluation\else การประเมินระบบ\fi
\fi}

ในบทนี้จะทดสอบเกี่ยวกับการทำงานในฟังก์ชันหลักๆ


\section{Objective (วัตถุประสงค์การทดสอบ)}
\begin{enumerate}
    \item ตรวจสอบความสามารถของโมเดล CNN-LSTM ในการ แยกแยะสถานะ normal และ anomaly จาก SWaT dataset
    \item วัดประสิทธิภาพของโมเดลตามตัวชี้วัดมาตรฐาน:
    \begin{enumerate}
        \item Precision, Recall, F1-score, AUC
        \item Detection Delay และ Latency สำหรับการทดสอบแบบ real-time
    \end{enumerate}
    \item ตรวจสอบ robustness ของโมเดลต่อ:
    \begin{enumerate}
        \item Noise (sensor fluctuation)
        \item Missing data
        \item Concept drift หรือ pattern ใหม่ที่ไม่เคยเห็น
    \end{enumerate}
    \item เปรียบเทียบ performance กับ baseline models เช่น LSTM, CNN, Autoencoder
\end{enumerate}

\section{Requirements (ข้อกำหนดการทดสอบ)}
\subsection{Functional Requirements (ข้อกำหนดเชิงฟังก์ชัน)}
\begin{enumerate}
    \item โมเดลต้องสามารถจำแนก Normal vs Anomaly สำหรับข้อมูล time-series
    \item สามารถแจ้งเตือน (Alert) เมื่อพบ anomaly ได้แบบ real-time หรือ batch
    \item รองรับ input จาก sensor/actuator ตาม SWaT dataset
    \item มี visualization/logging ของผลลัพธ์และ reconstruction error (ถ้ามี)
\end{enumerate}

\subsection{Non-Functional Requirements (ข้อกำหนดไม่เชิงฟังก์ชัน)}
\begin{enumerate}
    \item Accuracy: F1-score $\ge$ 85\% (ตัวอย่าง benchmark)
    \item Latency: การทำนาย ≤ 20 ms ต่อ window (real-time)
    \item Robustness: Performance degradation ≤ $\ge$ 10\% เมื่อมี noise/missing data
    \item Scalability: รองรับจำนวน sensor ≥ 50 และ batch size ≥ 64
    \item Usability: รายงานผลลัพธ์ในรูปแบบตาราง/กราฟเข้าใจง่าย
\end{enumerate}

\subsection{Dataset Requirements}
\begin{enumerate}
    \item ใช้ SWaT dataset: Normal logs สำหรับ train, Normal + Attack logs สำหรับ test
    \item แบ่งข้อมูลเป็น training/validation/test sets อย่างเหมาะสม
    \item Preprocessing: normalization, windowing, imputation for missing data
\end{enumerate}

\section{Testing Strategy (กลยุทธ์การทดสอบ)}
\subsection{Types of Testing}
\begin{enumerate}
    \item Functional Testing:
    \begin{enumerate}
        \item ตรวจสอบว่าโมเดลสามารถระบุ anomaly ใน test set ได้ถูกต้อง
        \item ตรวจสอบ alert system สำหรับ online detection
    \end{enumerate}
    \item Performance Testing:
    \begin{enumerate}
        \item ประเมิน Precision, Recall, F1-score, AUC
        \item วัด latency และ throughput
    \end{enumerate}
    \item Robustness Testing:
    \begin{enumerate}
        \item เพิ่ม noise ในข้อมูล sensor
        \item ลอง missing values (random drop)
        \item ตรวจสอบ model performance ต่อ concept drift หรือ attack ใหม่
    \end{enumerate}\item Comparison Testing: \\ เปรียบเทียบกับ baseline models: LSTM, CNN, Autoencoder
\end{enumerate}