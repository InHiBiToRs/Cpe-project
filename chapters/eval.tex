\chapter{\ifproject%
\ifenglish Experimentation and Results\else การทดลองและผลลัพธ์\fi
\else%
\ifenglish System Evaluation\else การประเมินระบบ\fi
\fi}

\section{วัตถุประสงค์การทดสอบ (Objective)}
\hspace{2em} การประเมินผลของโครงงานนี้มีจุดประสงค์เพื่อวิเคราะห์ประสิทธิภาพของโมเดล CNN-LSTM \\ ที่พัฒนาขึ้นสำหรับการตรวจจับความผิดปกติ (Anomaly Detection) ใน SWaT dataset โดยมีวัตถุประสงค์สำคัญดังนี้:
\begin{enumerate}
    \item ตรวจสอบความสามารถของโมเดลในการจำแนกสถานะ ปกติ (Normal) และ ผิดปกติ (Anomaly) จากข้อมูล time-series ของ sensor และ actuator
    \item ประเมินประสิทธิภาพตามตัวชี้วัดมาตรฐาน ได้แก่ Precision, Recall, F1-score, Area Under Curve (ROC-AUC, PR-AUC) เพื่อวัดความแม่นยำเชิงสถิติ
    \item วิเคราะห์ Detection Delay และ Latency เพื่อพิจารณาความเหมาะสมในการใช้งานเชิงปฏิบัติการแบบ real-time
    \item ทดสอบความทนทาน (Robustness) ของโมเดลต่อเงื่อนไขที่อาจเกิดขึ้นจริง เช่น
    \begin{itemize}
        \item Noise จากความผันผวนของ sensor
        \item Missing data (ข้อมูลสูญหายระหว่างการเก็บ)
        \item Concept drift หรือ pattern ใหม่ที่ไม่เคยปรากฏมาก่อน
    \end{itemize}
    \item เปรียบเทียบผลลัพธ์กับโมเดล baseline เช่น CNN เดี่ยว, LSTM เดี่ยว, Autoencoder เพื่อยืนยันความเหมาะสมของ CNN-LSTM
\end{enumerate}

\section{Requirements (ข้อกำหนดการทดสอบ)}
\subsection{Functional Requirements (ข้อกำหนดเชิงฟังก์ชัน)}
\begin{enumerate}
    \item โมเดลต้องสามารถจำแนกข้อมูล Normal vs Anomaly ได้อย่างถูกต้อง
    \item ระบบต้องสามารถ แจ้งเตือน (Alert System) ได้ทั้งแบบ batch processing และ real-time detection
    \item รองรับการนำเข้าข้อมูลจาก sensor/actuator ที่มีโครงสร้างตาม SWaT dataset
    \item มีการจัดเก็บและนำเสนอผลลัพธ์ในรูปแบบ visualization และ log files (เช่น confusion matrix, error curves)
\end{enumerate}

\subsection{Non-Functional Requirements (ข้อกำหนดไม่เชิงฟังก์ชัน)}
\begin{enumerate}
    \item Accuracy Benchmark: ค่าคะแนน F1-score ต้องไม่ต่ำกว่า $\ge$ 85\% (อ้างอิง benchmark งานวิจัยที่เกี่ยวข้อง)
    \item Latency: การทำนายผลแต่ละ window ต้องใช้เวลา ≤ 20 ms เพื่อรองรับ real-time operation
    \item Robustness: ประสิทธิภาพต้องลดลงไม่เกิน $\ge$ 10\% ภายใต้เงื่อนไข noise หรือ missing data
    \item Scalability: รองรับจำนวน sensor อย่างน้อย 50 ตัว และ batch size ≥ 64
    \item Usability: ผลลัพธ์ต้องสามารถแสดงในรูป ตารางและกราฟ ที่เข้าใจง่ายต่อผู้ปฏิบัติการ
\end{enumerate}

\subsection{Dataset Requirements}
\begin{enumerate}
    \item ใช้ SWaT dataset โดยแบ่งเป็น Normal logs สำหรับการฝึกโมเดล และ Normal + Attack logs สำหรับการทดสอบ
    \item การแบ่งข้อมูลต้องเป็น Training / Validation / Testing sets โดยพิจารณาลำดับเวลาเพื่อป้องกัน data leakage
    \item ต้องทำการ Preprocessing เช่น normalization, sliding window, และการแทนค่าข้อมูลที่หายไป (imputation)
\end{enumerate}

\section{กลยุทธ์การทดสอบ (Testing Strategy)}
\subsection{ประเภทของการทดสอบ (Types of Testing)}
\begin{enumerate}
    \item Functional Testing
    \begin{enumerate}
        \item ตรวจสอบความถูกต้องในการจำแนก anomaly ใน test set
        \item ทดสอบการทำงานของ alert system ในสภาพ online detection
    \end{enumerate}
    \item Performance Testing
    \begin{enumerate}
        \item ประเมินผลด้วย Precision, Recall, F1-score, ROC-AUC และ PR-AUC
        \item วัด latency และ throughput ของระบบ
    \end{enumerate}
    \item Robustness Testing
    \begin{enumerate}
        \item เพิ่ม noise ลงในข้อมูล sensor และวัดผลลัพธ์
        \item จำลองการ missing values แบบสุ่มและวิเคราะห์ผลกระทบ
        \item ทดสอบ response ต่อ concept drift หรือการโจมตีที่ไม่เคยปรากฏใน training
    \end{enumerate}
    \item Comparison Testing
    \begin{itemize}
        \item เปรียบเทียบกับ baseline models ได้แก่ LSTM, CNN, Autoencoder
        \item วิเคราะห์ผลลัพธ์เพื่อยืนยันข้อดีของโมเดล CNN-LSTM
    \end{itemize}
\end{enumerate}