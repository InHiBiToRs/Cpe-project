\chapter{\ifenglish Background Knowledge and Theory\else ทฤษฎีที่เกี่ยวข้อง\fi}

การทำโครงงาน เริ่มต้นด้วยการศึกษาค้นคว้า ทฤษฎีที่เกี่ยวข้อง หรือ งานวิจัย/โครงงาน ที่เคยมีผู้นำเสนอไว้แล้ว ซึ่งเนื้อหาในบทนี้ก็จะเกี่ยวกับการอธิบายถึงสิ่งที่เกี่ยวข้องกับโครงงาน เพื่อให้ผู้อ่านเข้าใจเนื้อหาในบทถัดๆ ไปได้ง่ายขึ้น

\section{Operational Technology (OT) และ Industrial Control Systems (ICS)}

\hspace{2em} Operational Technology (OT) คือเทคโนโลยีที่ใช้สำหรับตรวจสอบและควบคุมกระบวนการทางกายภาพในโรงงานอุตสาหกรรมหรือโครงสร้างพื้นฐานที่สำคัญ เช่น ระบบไฟฟ้า ระบบน้ำมัน และระบบบำบัดน้ำเสีย จุดเด่นของ OT คือการทำงานที่ต้องเน้น ความต่อเนื่อง ความเสถียร และความปลอดภัย มากกว่า IT (Information Technology) ที่มุ่งเน้นการประมวลผลข้อมูลและธุรกรรมเป็นหลัก

\subsubsection*{Industrial Control Systems (ICS) เป็นกลุ่มของระบบที่ใช้ควบคุมและจัดการการทำงานของ OT โดยมีตัวอย่างที่สำคัญ ได้แก่}
\begin{enumerate}
  \item Supervisory Control and Data Acquisition (SCADA) ใช้สำหรับควบคุมและตรวจสอบระบบขนาดใหญ่ที่กระจายตัว
  \item Distributed Control System (DCS) ใช้ควบคุมกระบวนการแบบต่อเนื่อง เช่น โรงกลั่นน้ำมัน
  \item Programmable Logic Controller (PLC) ใช้ในกระบวนการที่ต้องการการควบคุมแบบเฉพาะกิจ
\end{enumerate}

\section{ความมั่นคงปลอดภัยไซเบอร์ในระบบ OT/ICS}

\hspace{2em} ระบบ ICS เดิมถูกออกแบบมาเพื่อเน้นความทนทานและการทำงานอย่างต่อเนื่อง โดยไม่ได้คำนึงถึงการป้องกันภัยไซเบอร์ ทำให้มีช่องโหว่ต่อการโจมตี เช่น:
\begin{enumerate}
  \item Stuxnet (2010): มัลแวร์ที่แทรกแซงการทำงานของ PLC ในโรงงานนิวเคลียร์อิหร่าน
  \item Ukraine Power Grid Attack (2015): การโจมตีระบบไฟฟ้าทำให้เกิดการดับไฟในวงกว้าง
\end{enumerate}
\indent
กรณีศึกษาดังกล่าวสะท้อนว่าภัยคุกคามไซเบอร์สามารถสร้างความเสียหายทั้งด้านเศรษฐกิจ ความมั่นคง และความปลอดภัยของประชาชน การพัฒนา ระบบตรวจจับความผิดปกติ (Anomaly Detection) จึงมีความสำคัญอย่างยิ่งใน OT

\subsection{Anomaly Detection}
\hspace{2em} Anomaly Detection คือกระบวนการระบุข้อมูลที่เบี่ยงเบนไปจากรูปแบบปกติ ซึ่งอาจเกิดจากการโจมตี ความผิดพลาดของอุปกรณ์ หรือความผิดปกติของกระบวนการ โดยทั่วไปแบ่งได้เป็น 3 ประเภท:

\begin{enumerate}
  \item Point Anomaly: ข้อมูลจุดเดียวที่ผิดปกติ (เช่น ค่า Sensor พุ่งสูงกว่าปกติ)
  \item Contextual Anomaly: ข้อมูลที่ปกติในบางบริบท แต่ผิดปกติในอีกบริบทหนึ่ง (เช่น ค่าน้ำสูงในฤดูฝนเป็นเรื่องปกติ แต่สูงในฤดูแล้งถือว่าผิดปกติ)
  \item Collective Anomaly: ลำดับข้อมูลหลายจุดที่รวมกันแล้วผิดปกติ (เช่น การทำงานของ Pump และ Valve ที่ไม่สัมพันธ์กัน)
\end{enumerate}
\indent
สำหรับระบบ ICS Collective Anomaly มีความสำคัญมาก เนื่องจากข้อมูลจาก Sensor และ Actuator อยู่ในรูปแบบ Time-Series ที่ต้องพิจารณาลำดับเวลา

\section{ Secure Water Treatment (SWaT) Dataset}
\hspace{2em} Secure Water Treatment (SWaT) Dataset เป็นชุดข้อมูลที่สร้างขึ้นจากโรงงานจำลองระบบบำบัดน้ำ (Water Treatment Testbed) โดย iTrust Lab, Singapore University of Technology and Design (SUTD) มีวัตถุประสงค์เพื่อใช้เป็น มาตรฐานกลาง (benchmark) สำหรับการวิจัยด้านความปลอดภัยไซเบอร์ในระบบควบคุมอุตสาหกรรม (ICS/SCADA)

\subsubsection{โครงสร้างของระบบ SWaT}
โรงงานจำลอง SWaT ออกแบบให้มีลักษณะใกล้เคียงโรงงานจริง โดยแบ่งออกเป็น 6 กระบวนการย่อย (Processes/Stages):

\begin{enumerate}
  \item P1: Raw Water Supply – ระบบสูบน้ำดิบเข้าสู่ถังเก็บ
  \item P2: Pre-treatment – การกรองเบื้องต้น (Ultrafiltration)
  \item P3: Chemical Dosing – การเติมสารเคมี เช่น กรด-ด่าง เพื่อปรับค่า pH
  \item P4: Membrane-based Ultra-Filtration – การกรองละเอียดด้วยเมมเบรน
  \item P5: Dechlorination – การกำจัดคลอรีนที่เหลืออยู่
  \item P6: Backwash and Product Storage – การล้างย้อนและการเก็บน้ำสะอาดในถังเก็บสุดท้าย
\end{enumerate}
\indent
ในแต่ละกระบวนการจะมีการติดตั้ง เซนเซอร์ (sensors) และ แอกชูเอเตอร์ (actuators) เช่น ปั๊ม (pumps), วาล์ว (valves), และเครื่องวัดระดับน้ำ (level sensors) ซึ่งควบคุมด้วย PLC (Programmable Logic Controllers) และเฝ้าติดตามผ่านระบบ SCADA

\subsubsection{ประเภทตัวแปรใน Dataset}
\hspace{2em} ข้อมูลที่บันทึกจาก SWaT เป็น Time-Series รายวินาที (1 sample/second) ครอบคลุมช่วงเวลา 11 วัน รวม 946,722 records และ 51 attributes

\subsubsection{ข้อมูลทางกายภาพ (Physical Properties)}
\begin{enumerate}
  \item Sensors (ตัวแปรต่อเนื่อง)
  \begin{enumerate}
    \item Flow (อัตราการไหล)
    \item Level (ระดับน้ำ)
    \item Pressure (ความดัน)
    \item Conductivity (ค่าการนำไฟฟ้า)
    \item pH (กรด-ด่าง)
    \item ORP (Oxidation-Reduction Potential)
    \item Temperature (อุณหภูมิ)
  \end{enumerate}
  \item Actuators (ตัวแปรไม่ต่อเนื่อง)
  \begin{enumerate}
    \item ปั๊ม (Pumps – เปิด/ปิด)
    \item วาล์ว (Motorized Valves – เปิด/ปิด)
    \item Dosing Pumps (สำหรับจ่ายสารเคมี)
  \end{enumerate}
\end{enumerate}

\subsubsection{สถานการณ์การโจมตี (Attack Scenarios)}
\hspace{2em} ในการทดลอง มีการออกแบบการโจมตีทั้งหมด 36 รูปแบบ แบ่งเป็น:

\begin{enumerate}
  \item Single Stage Single Point (SSSP) – โจมตีจุดเดียวในกระบวนการเดียว (26 ครั้ง)
  \item Single Stage Multi Point (SSMP) – โจมตีหลายจุดในกระบวนการเดียว (4 ครั้ง)
  \item Multi Stage Single Point (MSSP) – โจมตี 1 จุด แต่มีผลกระทบหลายกระบวนการ (2 ครั้ง)
  \item Multi Stage Multi Point (MSMP) – โจมตีหลายจุดในหลายกระบวนการ (4 ครั้ง)
\end{enumerate}

\subsubsection*{ตัวอย่าง:}
\begin{enumerate}
  \item ปลอมค่าจาก LIT101 (Level Sensor) ทำให้ระบบเข้าใจผิดว่าน้ำเต็มถัง และสั่งหยุดปั๊ม ส่งผลให้น้ำล้น (Overflow)
  \item ปลอมค่าจาก LIT301 ให้แสดงค่าสูงผิดปกติ ส่งผลให้ปั๊มทำงานต่อแม้น้ำหมด และเกิด Underflow และอาจทำให้ปั๊มเสียหาย
\end{enumerate}

\section{Machine Learning}
\hspace{2em} Machine Learning (ML) เป็นแขนงหนึ่งของปัญญาประดิษฐ์ (Artificial Intelligence: AI) ที่ช่วยให้ระบบสามารถเรียนรู้และปรับปรุงประสิทธิภาพจากข้อมูล โดยไม่ต้องเขียนกฎแบบตายตัวล่วงหน้า การป้อนข้อมูลจำนวนมากที่หลากหลายช่วยให้โมเดลสามารถสกัดรูปแบบ (patterns) และปรับปรุงความ \\ แม่นยำได้อย่างต่อเนื่อง

สำหรับโครงการนี้ ML ถูกนำมาใช้ในการตรวจจับความผิดปกติ (Anomaly Detection) ใน ข้อมูลเชิงเวลา (time-series) ที่ได้จาก ระบบควบคุมอุตสาหกรรม (ICS/OT) ซึ่งประกอบด้วยเซนเซอร์และแอกชูเอเตอร์ในระบบบำบัดน้ำ (SWaT dataset)

\section{Convolutional Neural Network – Long Short-Term Memory (CNN-LSTM)}
\hspace{2em} การตรวจจับความผิดปกติในข้อมูลเชิงเวลา (time-series anomaly detection) จำเป็นต้องอาศัยโมเดลที่สามารถสกัดคุณลักษณะเชิงลึกและในขณะเดียวกันต้องเข้าใจความสัมพันธ์เชิงลำดับเวลา โมเดล \\ CNN-LSTM จึงถูกนำมาใช้เพราะผสานข้อดีของ Convolutional Neural Network (CNN) และ Long Short-Term Memory (LSTM) เข้าด้วยกัน ทำให้สามารถจัดการกับข้อมูลจากระบบควบคุมอุตสาหกรรม (ICS) ที่มีความซับซ้อนสูงได้อย่างมีประสิทธิภาพ โดยเฉพาะข้อมูลจาก SWaT dataset ซึ่งประกอบด้วยข้อมูลจาก sensor และ actuator ที่บันทึกเป็นลำดับเวลา

\subsection{Convolutional Neural Network (CNN)}
\hspace{2em} CNN เป็นโครงข่ายประสาทเทียมที่พัฒนาขึ้นเพื่อตรวจจับและสกัดคุณลักษณะสำคัญจากข้อมูล โดยอาศัยการทำงานของ convolution filters ที่เลื่อนผ่านข้อมูลเพื่อค้นหารูปแบบซ้ำหรือความเปลี่ยนแปลงที่มีนัยสำคัญ ในบริบทของข้อมูลเชิงเวลา CNN สามารถทำหน้าที่ตรวจสอบการเปลี่ยนแปลงของค่า sensor ในช่วงสั้น ๆ ได้อย่างมีประสิทธิภาพ เช่น การแกว่งของระดับน้ำหรือการเปลี่ยนแปลงค่าความดันอย่างเฉียบพลัน นอกจากนี้ CNN ยังช่วยลดสัญญาณรบกวน (noise) และเน้นคุณลักษณะที่บ่งบอกถึงความผิดปกติได้อย่างชัดเจน

\subsection{Long Short-Term Memory (LSTM)}
\hspace{2em} LSTM เป็นโครงข่ายประสาทเทียมแบบ Recurrent Neural Network (RNN) ที่ถูกออกแบบมาเพื่อแก้ไขข้อจำกัดด้านการจำข้อมูลระยะยาวของ RNN ทั่วไป โดยใช้โครงสร้างที่มีกลไกสำคัญคือ forget gate, input gate และ output gate เพื่อกำหนดว่าข้อมูลใดควรถูกเก็บรักษาไว้และข้อมูลใดควรถูกลืมไป ทำให้ LSTM สามารถเรียนรู้ความสัมพันธ์ที่ซับซ้อนและยาวนานของข้อมูลเชิงเวลาได้อย่างมีประสิทธิภาพ สำหรับข้อมูลจาก SWaT dataset LSTM สามารถตรวจสอบความเชื่อมโยงของ sensor และ actuator ที่ทำงานต่อเนื่องหลายขั้นตอน เช่น ลำดับการเปิดปิดของปั๊มและวาล์ว ซึ่งอาจเป็นปัจจัยสำคัญที่ทำให้เกิดความผิดปกติในกระบวนการ

\subsection{CNN-LSTM for Anomaly Detection}
\hspace{2em} การผสมผสาน CNN และ LSTM เข้าไว้ด้วยกันในโมเดล CNN-LSTM ทำให้สามารถใช้ CNN ในการสกัดคุณลักษณะสำคัญจากข้อมูลเชิงเวลาในช่วงสั้น แล้วส่งผลลัพธ์ต่อให้ LSTM เพื่อเรียนรู้ความสัมพันธ์ที่ต่อเนื่องในระยะยาว แนวทางนี้ช่วยให้โมเดลสามารถจับพฤติกรรมผิดปกติที่ซับซ้อนในระบบ ICS ได้ดียิ่งขึ้น เมื่อเปรียบเทียบกับการใช้ CNN หรือ LSTM เพียงอย่างเดียว งานวิจัยจำนวนมากยืนยันว่าโมเดล CNN-LSTM มีประสิทธิภาพในการลดอัตราการตรวจจับผิดพลาด (false positives) และเพิ่มความแม่นยำในการตรวจจับ anomaly ได้ โดยเฉพาะอย่างยิ่งเมื่อใช้กับข้อมูลที่มีลักษณะ time-series แบบหลายตัวแปร (multivariate time-series) เช่นใน SWaT dataset

\subsection{Dense Layer และ Fully Connected Layer}
\hspace{2em} หลังจากที่ CNN และ LSTM ทำหน้าที่สกัดคุณลักษณะและจับความสัมพันธ์เชิงเวลาแล้ว ผลลัพธ์ที่ได้จะถูกส่งต่อไปยัง Dense Layer หรือ Fully Connected Layer เพื่อทำการรวมข้อมูลและสร้างการตัดสินใจสุดท้าย Dense Layer ทำงานโดยเชื่อมต่อทุกนิวรอนกับนิวรอนในชั้นถัดไป ช่วยให้โมเดลสามารถเรียนรู้การรวมคุณลักษณะที่ซับซ้อนและสร้างการจำแนก (classification) ระหว่างเหตุการณ์ปกติและเหตุการณ์ผิดปกติได้อย่างมีประสิทธิภาพ
ในโครงการนี้ Dense Layer ทำหน้าที่แปลง representation ที่ได้จาก CNN-LSTM ให้กลายเป็นค่าความน่าจะเป็น (probability) ของ class เช่น Normal (0) และ Anomaly (1)

\subsection{Activation Functions}
\hspace{2em} ฟังก์ชันกระตุ้น (Activation Functions) เป็นองค์ประกอบสำคัญของ Dense Layer ซึ่งใช้ในการกำหนดเส้นแบ่งเชิงเส้นหรือไม่เชิงเส้น เช่น:
\begin{enumerate}
  \item ReLU (Rectified Linear Unit): ใช้ในชั้นซ่อน (hidden layers) ของ CNN และ LSTM เพื่อลดปัญหา vanishing gradient และเร่งการเรียนรู้
  \item Sigmoid: เหมาะกับการจำแนกแบบ binary anomaly detection เนื่องจากผลลัพธ์อยู่ระหว่าง 0–1
  \item Softmax: ใช้สำหรับ multi-class anomaly detection เมื่อมีความผิดปกติหลายประเภท
\end{enumerate}

\subsection{Regularization Layers (Dropout และ Batch Normalization)}
\hspace{2em} เพื่อป้องกันการเกิด overfitting จากข้อมูลที่มีความซับซ้อนสูง เช่น SWaT dataset การเพิ่ม Regularization Layers มีความสำคัญมาก ได้แก่:
\begin{enumerate}
  \item Dropout Layer: ทำการสุ่มปิดบางนิวรอนระหว่างการฝึก เพื่อป้องกันไม่ให้โมเดลจดจำข้อมูลมากเกินไป แต่ช่วยให้เกิดการเรียนรู้ที่ครอบคลุมมากขึ้น
  \item Batch Normalization: ช่วยปรับค่าการกระจายของข้อมูลในแต่ละชั้นให้อยู่ในช่วงที่เหมาะสม ทำให้การฝึกโมเดลเสถียรและรวดเร็วขึ้น
\end{enumerate}

\section{Correlation Matrix}
\hspace{2em} Correlation Matrix เป็นเครื่องมือเชิงสถิติที่ใช้วิเคราะห์ความสัมพันธ์ระหว่างตัวแปรหลายตัวพร้อมกัน โดยจะแสดงค่า Correlation Coefficient ที่มีค่าตั้งแต่ -1 ถึง 1
\begin{enumerate}
  \item ค่า 1 หมายถึงความสัมพันธ์เชิงบวกอย่างสมบูรณ์ (เมื่อค่าของตัวแปรหนึ่งเพิ่ม อีกตัวก็เพิ่มตาม)
  \item ค่า -1 หมายถึงความสัมพันธ์เชิงลบอย่างสมบูรณ์ (เมื่อค่าของตัวแปรหนึ่งเพิ่ม อีกตัวจะลดลง)
  \item ค่า 0 หมายถึงไม่มีความสัมพันธ์
\end{enumerate}
สำหรับ SWaT dataset การสร้าง Correlation Matrix มีความสำคัญในหลายประเด็น ได้แก่
\begin{enumerate}
  \item การทำความเข้าใจความสัมพันธ์ของ Sensor และ Actuator: เช่น ระดับน้ำ (Level) ควรมีความสัมพันธ์กับการทำงานของปั๊ม (Pump) และวาล์ว (Valve)
  \item การตรวจหาความผิดปกติ (Anomaly): หาก correlation ระหว่าง sensor และ actuator ไม่เป็นไปตามปกติ เช่น ค่า Flow ไม่สัมพันธ์กับการเปิดปิด Valve อาจบ่งชี้ถึงการโจมตีหรือความผิดปกติในกระบวนการ
  \item Feature Selection: การวิเคราะห์ความสัมพันธ์ช่วยคัดเลือกตัวแปรที่สำคัญ และลด \\ multicollinearity เพื่อลด noise ก่อนนำเข้าสู่โมเดล Machine Learning
\end{enumerate}
\indent
การใช้ Correlation Matrix จึงเป็นขั้นตอนสำคัญใน Exploratory Data Analysis (EDA) ของโครงการนี้ เพื่อทำให้การออกแบบและฝึกสอนโมเดล anomaly detection มีความถูกต้องและแม่นยำมากขึ้น

\section{Model Evaluation}
การประเมินผลลัพธ์ของโมเดลมีความสำคัญอย่างยิ่ง เนื่องจากเป็นวิธีการตรวจสอบความถูกต้องและความน่าเชื่อถือของโมเดลในการตรวจจับความผิดปกติจากข้อมูลจริง โดยเฉพาะอย่างยิ่งในกรณีของ SWaT dataset ซึ่งมีลักษณะข้อมูล ไม่สมดุล (imbalanced data) กล่าวคือ ข้อมูลเหตุการณ์ปกติ (Normal) มีจำนวนมากกว่าข้อมูลเหตุการณ์ผิดปกติ (Anomaly) อย่างมาก การใช้ตัวชี้วัด (Evaluation Metrics) ที่เหมาะสมจึงมีบทบาทสำคัญในการสะท้อนศักยภาพของโมเดลได้อย่างครบถ้วน
\\ ตัวชี้วัดที่ใช้ประกอบการประเมินในโครงการนี้ ได้แก่:
\begin{enumerate}
\item Accuracy \\ เป็นสัดส่วนของจำนวนข้อมูลทั้งหมดที่โมเดลสามารถจำแนกได้ถูกต้อง แต่เนื่องจากข้อมูลมีความ \\ ไม่สมดุล ค่าความแม่นยำ (accuracy) เพียงอย่างเดียวอาจไม่สะท้อนศักยภาพของโมเดลได้อย่างแท้จริง เพราะโมเดลที่ทำนายว่าข้อมูลเป็น “ปกติ” ตลอดเวลาอาจได้ค่า accuracy สูง แต่ไม่สามารถตรวจจับ anomaly ได้เลย
\item Precision \\ เป็นตัวชี้วัดความถูกต้องของการทำนาย anomaly หมายถึง ในจำนวนทั้งหมดที่โมเดลทำนายว่า “ผิดปกติ” มีสัดส่วนเท่าใดที่เป็น anomaly จริง Precision ที่สูงบ่งชี้ว่าโมเดลสามารถลดการแจ้งเตือนผิดพลาด (False Positives) ได้
\item Recall (Sensitivity) \\ เป็นตัวชี้วัดความสามารถของโมเดลในการตรวจจับ anomaly ได้ครบถ้วน หมายถึงในจำนวน \\ anomaly ทั้งหมดที่มีอยู่โมเดลสามารถตรวจพบได้กี่กรณี Recall จึงมีความสำคัญอย่างยิ่งในงาน ICS/OT anomaly detection เพราะการพลาดการตรวจจับ anomaly เพียงเล็กน้อยอาจก่อให้เกิดผลกระทบต่อความปลอดภัยและการดำเนินงานของระบบ
\item F1-score \\ เป็นค่าเฉลี่ยเชิงฮาร์โมนิก (Harmonic Mean) ระหว่าง Precision และ Recall ทำให้สะท้อนสมดุลระหว่างการลดการแจ้งเตือนผิดพลาดและการเพิ่มอัตราการตรวจจับ anomaly ได้อย่างเหมาะสม โดยเฉพาะอย่างยิ่งใน dataset ที่มี class imbalance
\item Confusion Matrix \\ เป็นตารางที่แสดงการจำแนกผลลัพธ์ของโมเดลอย่างละเอียดในรูปของ True Positive (TP), True Negative (TN), False Positive (FP) และ False Negative (FN) ช่วยให้สามารถวิเคราะห์ข้อผิดพลาดของโมเดลได้เชิงลึก และชี้ชัดว่าปัญหาของโมเดลอยู่ที่การตรวจจับ anomaly ไม่เพียงพอ (Recall ต่ำ) หรือการแจ้งเตือนผิดพลาดสูง (Precision ต่ำ)
\item ROC-AUC (Receiver Operating Characteristic – Area Under Curve) \\ เป็นการประเมินสมรรถนะของโมเดลโดยไม่ขึ้นกับ threshold โดยใช้ค่า True Positive Rate และ False Positive Rate มาวิเคราะห์เส้นโค้ง (ROC Curve) ค่า AUC ที่สูงใกล้ 1 บ่งชี้ว่าโมเดลสามารถจำแนก anomaly ออกจาก normal ได้อย่างมีประสิทธิภาพ
\item PR-AUC (Precision-Recall Area Under Curve) \\ เป็นการวัดสมรรถนะที่เหมาะสมอย่างยิ่งสำหรับข้อมูลที่ไม่สมดุล (imbalanced dataset) โดยแสดงความสัมพันธ์ระหว่าง Precision และ Recall ตลอดทุก threshold ค่าที่สูงใกล้ 1 แสดงว่าโมเดลมีความสามารถในการตรวจจับ anomaly ได้ดีแม้ในสภาพที่ anomaly มีสัดส่วนค่อนข้างน้อย
\end{enumerate}