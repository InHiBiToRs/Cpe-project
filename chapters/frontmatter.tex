\maketitle
\makesignature

\ifproject
\begin{abstractTH}
\hspace{2em} ในยุคอุตสาหกรรม 4.0 โรงงานจำนวนมากได้ปรับใช้ ระบบควบคุมอัตโนมัติ (Industrial Control Systems: ICS) ที่เชื่อมโยงกับ เทคโนโลยีปฏิบัติการ (Operational Technology: OT) เพื่อตรวจวัดและควบคุมกระบวนการผลิตแบบเรียลไทม์ อย่างไรก็ตาม ความเชื่อมโยงดังกล่าวก่อให้เกิดความเสี่ยงจากภัยคุกคามทางไซเบอร์ ซึ่งอาจสร้างผลกระทบต่อ เศรษฐกิจ ความปลอดภัยของบุคลากร และสิ่งแวดล้อม

โครงงานนี้มีวัตถุประสงค์เพื่อพัฒนา ต้นแบบระบบตรวจจับความผิดปกติ (Anomaly/Intrusion Detection System) โดยใช้เทคนิค การเรียนรู้เชิงลึก (Deep Learning) ที่มุ่งเน้นการวิเคราะห์ ข้อมูลลำดับเวลา (time-series data) ของเซนเซอร์และแอกชูเอเตอร์ในระบบ OT

กระบวนการดำเนินงานประกอบด้วย การจัดเตรียมและทำความสะอาดข้อมูล (Preprocessing), การออกแบบคุณลักษณะ (Feature Engineering), การสร้างหน้าต่างเวลา (Sliding Window) และการ \\ พัฒนาโมเดลตรวจจับความผิดปกติ โดยใช้สถาปัตยกรรม 1D CNN-LSTM ซึ่งสามารถเรียนรู้ได้ทั้ง ความสัมพันธ์เชิงเวลา (temporal dependencies) และ ความสัมพันธ์เชิงลักษณะ (spatial/feature dependencies) ของข้อมูลเซนเซอร์

ผลลัพธ์ของโมเดลถูกประเมินด้วยตัวชี้วัดมาตรฐาน ได้แก่ Accuracy, Precision, Recall, F1-score รวมถึงการวิเคราะห์เชิงลึกเพื่อยืนยันความสามารถของโมเดลในการจำแนก เหตุการณ์ปกติ และ เหตุการณ์โจมตี (Anomaly events) ได้อย่างมีประสิทธิภาพ

\end{abstractTH}

\begin{abstract}
\hspace{2em} In the era of Industry 4.0, many factories have adopted Industrial Control Systems (ICS) integrated with Operational Technology (OT) to monitor and control production processes in real time. However, this interconnection introduces potential cybersecurity risks, which may significantly impact the economy, personnel safety, and the environment.

This project aims to develop a prototype anomaly/intrusion detection system using Deep Learning techniques, focusing on the analysis of time-series data obtained from sensors and actuators within OT systems.

The research methodology includes data preprocessing, feature engineering, sliding window generation, and the design of an anomaly detection model based on the 1D CNN-LSTM architecture. This hybrid approach enables the model to capture both temporal dependencies and feature-level (spatial) correlations within sensor data.

The performance of the proposed model is evaluated using standard metrics such as Accuracy, Precision, Recall, and F1-score, along with further analysis to verify its capability in effectively distinguishing between normal operations and attack/anomaly events. The outcome of this study contributes to enhancing the cyber resilience of ICS/OT systems and provides a foundation for practical applications in industrial cybersecurity.

\end{abstract}

\iffalse
\begin{dedication}
This document is dedicated to all Chiang Mai University students.

Dedication page is optional.
\end{dedication}
\fi % \iffalse

%\begin{acknowledgments}
%Your acknowledgments go here. Make sure it sits inside the
%\texttt{acknowledgment} environment.

%\acksign{2020}{5}{25}
%\end{acknowledgments}%
\fi % \ifproject

\contentspage

\ifproject
\figurelistpage

\tablelistpage
\fi % \ifproject

% \abbrlist % this page is optional

% \symlist % this page is optional

% \preface % this section is optional
