\maketitle
\makesignature

\ifproject
\begin{abstractTH}
ในยุคอุตสาหกรรม 4.0 โรงงานจำนวนมากได้ปรับใช้ระบบควบคุมอัตโนมัติ (Industrial Control Systems: ICS) ที่เชื่อมโยงกับเทคโนโลยีปฏิบัติการ (Operational Technology: OT) เพื่อตรวจวัดและควบคุมกระบวนการผลิตแบบเรียลไทม์ อย่างไรก็ตาม ความเชื่อมโยงดังกล่าวได้นำมาซึ่งความเสี่ยงจากภัยคุกคามทางไซเบอร์ที่อาจสร้างผล กระทบทั้งในด้านเศรษฐกิจ ความปลอดภัย และสิ่งแวดล้อม โครงงานนี้มีวัตถุประสงค์เพื่อพัฒนา ต้นแบบระบบตรวจจับความผิดปกติ (Intrusion/Anomaly Detection) ที่ใช้เทคนิคการเรียนรู้เชิงลึก (Deep Learning) โดยมุ่งเน้นไปที่การวิเคราะห์ข้อมูลลำดับเวลา (time-series data) ของเซนเซอร์และแอกชูเอเตอร์ในระบบ OT

กระบวนการวิจัยประกอบด้วยการจัดเตรียมและทำความสะอาดข้อมูล (preprocessing) การออกแบบคุณลักษณะ (feature engineering) การสร้างหน้าต่างเวลา (sliding window) และการฝึกโมเดลตรวจจับความผิดปกติ โดยใช้ สถาปัตยกรรม CNN-LSTM แบบหนึ่งมิติ (1D CNN-LSTM) ซึ่งสามารถจับทั้งความสัมพันธ์เชิงเวลาและความสัมพันธ์เชิงลักษณะระหว่างเซนเซอร์ได้ ผลลัพธ์ของโมเดลถูกประเมินด้วยตัวชี้วัด เช่น Accuracy, Precision, Recall และ F1-score เพื่อตรวจสอบประสิทธิภาพในการแยกเหตุการณ์ปกติและเหตุการณ์โจมตี
\end{abstractTH}

\begin{abstract}
In the era of Industry 4.0, many industrial plants have adopted Industrial Control Systems (ICS) integrated with Operational Technology (OT) to monitor and control production processes in real time. However, this interconnection has also introduced significant cybersecurity risks that may cause severe impacts on economic stability, safety, and the environment. This project aims to develop a prototype anomaly and intrusion detection system leveraging Deep Learning techniques, with a focus on analyzing the time-series data of sensors and actuators within OT environments.

In the era of Industry 4.0, many industrial plants have adopted Industrial Control Systems (ICS) integrated with Operational Technology (OT) to monitor and control production processes in real time. However, this interconnection has also introduced significant cybersecurity risks that may cause severe impacts on economic stability, safety, and the environment. This project aims to develop a prototype anomaly and intrusion detection system leveraging Deep Learning techniques, with a focus on analyzing the time-series data of sensors and actuators within OT environments.
The research process involves several steps: data cleaning and preprocessing, feature engineering, sliding window generation, and training anomaly detection models. A 1D CNN-LSTM architecture is employed to capture both temporal dependencies and cross-sensor feature relationships. The model’s performance is evaluated using metrics such as Accuracy, Precision, Recall, and F1-score to assess its ability to distinguish between normal operations and cyberattacks.
\end{abstract}

\iffalse
\begin{dedication}
This document is dedicated to all Chiang Mai University students.

Dedication page is optional.
\end{dedication}
\fi % \iffalse

%\begin{acknowledgments}
%Your acknowledgments go here. Make sure it sits inside the
%\texttt{acknowledgment} environment.

%\acksign{2020}{5}{25}
%\end{acknowledgments}%
\fi % \ifproject

\contentspage

\ifproject
\figurelistpage

\tablelistpage
\fi % \ifproject

% \abbrlist % this page is optional

% \symlist % this page is optional

% \preface % this section is optional
